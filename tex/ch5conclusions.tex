\chapter{Conclusions}\label{ch:5conclusions}

After terminating the empirical analysis of two sectors in the world trade network, let me proceed by recalling what was the goal of this research thesis and how the proposed work aligns with that.
Given the datasets that I started with, the purpose was to extract from them a suitable set of structured data to easily support the construction of the trade networks. I had to overcome some difficulties and to adapt the nomenclatures and currencies in order to properly merge the information from Eurostat and from WTO. Once I achieved this intermediate step, the next task was to explore the vast literature of Social Network Theory and to choose the most appropriate techniques for studying this kind of weighted directed graphs, in which not all graph theory concepts (such as the path between two nodes) could find an interpretation and be applied. The statistical techniques that I evaluated can be summarized as a group of graph and node properties, measured through properly defined metrics which highlight different aspects of the network structure, and then the two algorithms (Stochastic Block Model and Louvain Method) that I overviewed and applied for the task of community detection. Finally, once the structured normalized dataset was built, and the statistical methods were studied, I wanted to combine the two and apply the reviewed techniques to some real-life scenarios like the world trade sectors of \textit{Food Products} and \textit{Crude petroleum and natural gas}. What I found out in the end is that, taken one by one, the methods of network analysis may have some shortcomings in capturing the full picture, but if instead they are analyzed together with the available metrics and algorithm output, they are quite powerful in providing us with an insightful and newer understanding of the global patterns in international trade.
Therefore, we can conclude that given a dataset of historical records of bilateral trade relations among countries, it is possible to build a methodology to begin studying the data using a tool set of descriptive statistics and network science methods. Such a step is of paramount importance in whatever statistical analysis one wants to conduct on a similar dataset, which can be either for modelling purposes or for the forecasting of policies and scenarios. 
However, such statistical procedures have their limitations: for example, when presented with two different methods for the same task, as in the case of community detection, one needs to be able to discern which parameter setting and which outcome is best representing the patterns and dynamics of the network. In order to do so, the researcher often needs to integrate the insights from the model with his knowledge from other sources, such as a domain knowledge of the renowned relationships or monopolistic roles that are present in the world trade network.
Building upon this research, one may also want to take a step further and think about possible developments of the methodology to account for more complex layers of interactions in the networks. For example, one could study in deep the temporal component that is present in these datasets, and understand how the sequence of graphs behaves when evolving through time: this is the branch of research which deals with dynamic network and their analysis. In alternative, one could try to expand on the other relevant dimension of this dataset after time, that is the sector categories: it is possible in fact to construct multi-layer networks, in which a couple of nodes can be joined by more than one edge, in this case for example, they would be representing the groups of trade products that are exchanged, or perhaps the different means of transport that countries use to trade goods and services.

% RECAP THE GOAL
% ANSWER THE RESEARCH QUESTIONS
%   RECAP STEP TESI
% RESULTS USEFUL
% LIMITATION & DEVELOPMENTS
    % dynamic networks
    % multi-layer networks

