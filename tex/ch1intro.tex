\chapter{Introduction and Literature Review}

Global trade refers to the exchange of capital, good or services across international territories, when there is the need for products that a country is not able to obtain domestically. The commerce between countries is very complex, as it is influenced by several factors, such as, for example, geographical barriers, government policies, laws, currencies and markets. Indeed, international trade has always been an interesting topic for scholars of different academic fields, from the legal to the economic and geopolitical. To facilitate the growth of international trade and make the processes smoother, some international economic organizations were formed, regulating trade relationships, the largest being the World Trade Organization (WTO), established in 1995. According to WTO's statistics \cite{wto2022stats}, world trade is increasing over time, and today's volume is about 40 times the level recorded in the 1950s. In order to ensure transparency, international organizations take care of providing access to trade data, which can be used by governments, corporations and trade associations to serve different purposes. In particular, they can be used to monitor the markets and study supply and demand, both for the purpose of analysis and for seeking suppliers. Also, these data can reveal industry trends, and unveil the reaction of the markets to historical events and political interventions.

\section{The role of Network Analysis}

In this research thesis, I am going to explore a combined dataset of world trade statistics through the means of Social Network Analysis: use visualization techniques to gather qualitative insights about the World Trade Network, defining and describing the topology of the network, producing and discussing the mainly used network statistics and their evolution in time, and then assessing specific topic which can be studied using the tools of Graph Theory.
One may naturally wonder what can a network approach add to our economic understanding of the dynamics of international trade, instead of other descriptive statistics methods generally used in the study of international trade. In brief, one could answer that network are all about \textit{relations}: their fundamental piece of information is not the single component but the \textit{dyad}, that is the relationship between two components. Furthermore, all these relationships are not analyzed by themselves, but always together with the set of all other relationships among all dyads. The implication is that the relation between two entities in a graph cannot be considered independent of the other relations, but instead it is \textit{interdependent}.
An answer to this question was provided by \textcite{fagiolo2010evolution} in their research paper about world trade: they claim that "this methodology allows for a better description of the existing heterogeneity in the degrees of connectivity and, hence, of international economic integration". What matters for a country to be integrated into the WTN is not only how much a country trades, but also the specific distribution of trade volumes across trading partners. The tools and metrics of network theory allow us to characterize important features of the WTN such as the number of trade partners of each country, whether its partners trade with each other and the level of dependency of the entire network on a specific country or a group of them. 


\section{Research aims and relevance}
% What is this work about
% What is significant and unique about this study
% why is this important and / or interesting? Why should this thesis be read?
% important for istat ??

The inspiration for this work comes from the research internship I did with ISTAT (Italian National Institute of Statistics) during six months of 2022. My work is based on a previous project developed by them, named \textit{Cosmopolitics} \cite{bruno2021european}: it consisted of a web tool to provide useful statistics and graph visualizations of the European commerce network\footnote{
    The tool is online at \href{https://cosmo.statlab.it/}{cosmo.statlab.it}; the code is available here \url{https://github.com/istat-methodology/cosmopolitics}}.
In particular, ISTAT was interested in conducting an analysis on international trade relations both at a macro and micro level: on the one hand, they wanted to represent trade relations as graphs and conduct scenario analysis, while on the other they aimed at producing descriptive statistics representing the trade situation between two countries in the period before and after a specific reference date (such as, for example, the COVID pandemic outbreak).
ISTAT used a publicly available dataset named COMEXT by \textcite{eurostat2022comext}, which contains, among other things, historical data about world imports and exports reported by EU countries, from 1988 until 2022.
% what they asked me
The intent of my research internship was to deeply explore this dataset, and inspect what insights and patterns may be found using the latest techniques of graph analysis on these data. Supported by them, I started to investigate how the data could be obtained, how I could construct an automatic method for collecting all the relevant materials, and how I could store them in order for me to retrieve them easily. 
% what i found missing on the data
After this phase, I continued with a thorough exploration of the available data, and I soon discovered that this dataset alone wasn't enough for the goals that we had in mind. In fact, COMEXT is constructed by Eurostat using the data that it receives from EU member countries, thus all the information about exchanges between two non-EU countries was missing. 
% how i fixed it
To compensate for this, I searched for a second source of data, and I found the open data provided by the World Trade Organization. This second dataset was appropriate to compensate the lacking part of COMEXT and allowed me to have complete information about world trade.
% and produced a methodology to visualize and get insights from the data, before applying models
Therefore, the core idea of the research I've conducted is to construct a combined dataset about world imports and exports of goods among countries, and then to produce a methodology to analyze, visualize and get insights from these data, which is of vital importance before one could experiment with modelling or forecasting methods. In particular, I succeeded in using graph theory concepts and methods to represent the world trade network as a directed graph, to compute relevant metrics about it and to make observations about the complex relationships among countries and how they relate to real world effects.
The outcome of this research is a theoretic overview of the appropriate methods of network science and their application to conduct and empirical analysis of trade sectors in the WTN. Such a research work is of interest for ISTAT because of its primary mission. In fact, ISTAT is committed to producing and communicating high-quality statistical information, analyses and forecasts, in order to develop detailed knowledge of Italy’s environmental, economic and social dimensions at various levels of geographical detail and to assist all members of society in decision-making processes.

\section{Previous works}

The corpus of statistical research on the world trade network (WTN) is vast, and it comprises many methods and tools that have been applied to the task. The first example that we can consider is a research article by \textcite{fagiolo2010evolution} in which the authors employ a weighted network approach to study the empirical properties of the web of trade relationships among world countries, and its evolution over time. What they were able to show using the data at hand was that most countries are characterized by weak trade links, but a small group of countries instead featured numerous strong partnerships. Their findings suggested that the WTN had the structure of a \textit{core-periphery} network, showing that the more connected countries had trade relations both with poorly connected ones and with a dense cluster of nations. Another approach to the topic has been provided by \textcite{barbieri2009conflict}, in which the nature of the relationship between trade and conflict is explored, and some conclusions are drawn about trade's relationship and impact on international conflict. They identified what were the most serious problems that scholars face when dealing with official trade statistics, and they propose a way to overcome them.
Moving on, \textcite{kaluza2010complex} choose to focus on a similar but different aspect of the picture, which is global cargo ship movements. With $90\%$ of the world trade being carried by sea, the maritime network of transportation is a fundamental piece in the puzzle of modern countries' economy. Instead of constructing a network with countries as nodes, they used ports and connected them with links according to the trading routes, thus regarding it as a transportation network. Their purpose was to improve the older assumptions based on gravity models of ship movements and take a step forward in understanding the patterns of global trade and bioinvasion. In 2011, an article written by \textcite{benedictis2011world} reported that they observed strong and increasing heterogeneity in countries' choice of trade partners and that their analysis showed that trade policies play a role in shaping the world trade network, such as the fact that WTO members are more closely connected than the rest of the world. 
Going further with the literature review, I found a paper whose research was very similar in structure and motifs to the one that I propose here, and this was by \textcite{benedictis2014bacicepii}. Their research consisted in an overview of the visualization techniques for world trade networks, and then a description of the most commonly used metrics to understand the topology and characteristics of each network. They then applied these methods to a number of specific trade categories. Although my research is similar to what they proposed, it has some differences, and it takes some steps further: the categories of traded products that I consider are more comprehensive and thus the results obtained are more generalizable, and also I add a part on community detection which was not present in their work.
Looking further into the pool of research which I reviewed and took inspiration from, we find a paper by \textcite{deguchi2014hubs}, in which the metrics of Page Rank and Hyperlink Induced Topic Search (HITS) are constructed for a weighted directed network such as the one under study here. What they suggest is to look at Hub and Authority values, since countries with large authority values have significant imports from large hub countries, while hub values are large for countries with significant exports to high-authority countries.
Similar to the concept of World Trade Network, we can look at the idea of the World Input-Output Network (WION), as it was studied by \textcite{cerina2015world}: this is constructed by viewing the global I/O system as an interdependent network where the nodes are the individual industries in different economies and the edges are the monetary good flows between industries. Their findings are that at the global level, industries are highly but asymmetrically connected, which implies that micro shocks can lead to macro fluctuations, while at the regional level they found that the world production is still operated nationally or at most regionally as the communities that they detect are either individual economies or geographical regions.
Lastly, a more recent research by \textcite{monken2021graph} aims at using state-of-the-art tools of Artificial Intelligence and apply them to the network of international trade: in their paper they introduce a method that measures causal scenarios during outlier events using neural networks, and they call it the Artificial Intelligence Network Explanation of Trade. This tool tailors AI techniques specifically for bilateral trade modeling, and it can be applied to network-like real-life structures such as global trade.


\paragraph{Chapter structure}

The structure of the next chapters is as follows. First, in Chapter \ref{ch:2data}, I will describe the main datasets I used and the procedure I followed to combine them into a unique dataset containing the information about international import and exports. I will guide the reader from the raw data to the creation of graphs that can be used for the analysis. Then, Chapter \ref{ch:3methods} is structured as a detailed description of the statistical techniques that can be applied to graph structures, providing the reader with the theoretical background needed to be able to perform a network analysis. In particular, I will formally define a graph, its main properties, and more sophisticated methods that are particularly useful to identify patterns in social networks. I will also compare different visualization strategies to show what kind of insights one is able to gather from each of them. Later, in Chapter \ref{ch:4scenarios}, I will apply the described techniques to the graphs representing the trade networks of two interesting categories of goods, \textit{Food Products} and \textit{Crude petroleum and natural gas}, investigating their structures and spotting patterns corresponding to real-life dynamics. Lastly, in Chapter \ref{ch:5conclusions} I will mention some potential research directions one can undertake from this thesis. 

MENTION THE CODE AND THE DATA ON GITHUB