\chapter{Introduction chapter}

ISTAT used a publicly available dataset (COMEXT, by Eurostat) which contains historical data about world imports and exports of goods since 2010, to gather insights and in the end produce a web tool to provide useful statistics and a graph visualization of the commerce network. 

The core idea of the research I've conducted is to construct a dataset about world import and exports of goods among world countries, to represent it as a directed graph and to produce insights about world trade using mathematical and statistical tools.
The inspiration for this work comes from the research internship I did with ISTAT (Italian National Institute of Statistics) and their previous research named \textit{Cosmopolitics}. % CITATION 
They used a publicly available dataset named COMEXT by \textcite{eurostat2022comext}, which contains, among other things, historical data about world imports and exports reported by EU countries, from 1988 until 2022. The project that they developed consisted in a web tool to provide useful statistics and a graph visualization of the European commerce network.\footnote{Tool online at  \href{https://cosmo.statlab.it/}{cosmo.statlab.it} Code available here \url{https://github.com/istat-methodology/cosmopolitics}}


\section{Research questions}

Now they are wondering what else could they exploit these data for:
\begin{itemize}
    \item Can we use the graph representation of the good exchanges to understand the commerce network and look at useful statistics of the graph, and how they relate to real world effects?
    \item[OLD] \textit{Starting from the graph representation of the commerce data, is it possible to train ML and NN methods (like GraphNN and Bayes Nets) to make predictions, evaluate scenarios and draw inferences on them?}
    \item Can we be able to build a model capable of learning how did commerce evolve through time and given the model be able to observe what changes happen in the system when we add, remove or modify a node or an edge?
    \item In fact, their research questions for my project are about using these data and the graphs generated to investigate whether it’s possible to apply ML and NN methods (like Bayesian Nets or GraphNNs) to draw inferences about the graphs and evaluate scenarios or compute forecasts.
\end{itemize}