\chapter{Data and Graphs chapter}

\section{Data Sources}

The data needed for constructing the dataset and performing the analysis come from multiple sources:
\begin{itemize}
    \item European Statistical Office (Eurostat) \cite{eurostat2022comext}
    \item World Trade Organization (WTO) \cite{wto2022stats}
    \item United Nations (UN) \cite{un2022population}
\end{itemize}

\subsection{COMEXT}

The starting dataset of my research is named COMEXT, and is published and maintained by Eurostat \cite{eurostat2022comext}. COMEXT is a statistical database for detailed statistics on international trade in goods (the definition of \textit{goods} is "all movable property including electricity"). It serves as an important indicator of the performance of the European Union (EU) economy, because it focuses on the size and the evolution of imports and exports of countries. It provides access not only to both recent and historical data of the EU and its individual Member States, but also to statistics of a significant number of non-EU countries. The information contained in it is based on data provided by the statistical agencies of the EU member states and trading partners.
Data are organized in tables, one for each year, and each table contains information about the country that declared the transaction, the partner country, the product that was exchanges according to multiple international standard classifications, the value in euros of the exchange and its quantity in kilograms.

\paragraph{COMEXT Data collection}
Historically, the main source of information about trade transactions between countries are customs authorities, which provide detailed information on exports and imports of goods with a geographical breakdown.
The COMEXT system was conceived and implemented in the early 90's,  following the adoption of the European Single Market on 1993, when customs formalities between Member States were removed. Since then, it has been continuously adapted and re-engineered to take into account technological evolution and new users' needs. The data gathered are based on two data collection systems:
\begin{itemize}
    \item data on trade in goods with non-EU countries are collected by customs authorities and are based on the records of trade transactions in customs declarations;
    \item data of intra-EU exchanges are directly collected from trade operators once a month.
\end{itemize}


\subsection{WTO}
Similar to Eurostat, also the World Trade Organization (WTO) collects data about the global commerce of products among countries \cite{wto2022stats}. Data are periodically sent to the organization from member countries, hence the dataset contains information about all the major world countries.

\section{Nomenclatures}

In order to classify products into categories, many nomenclatures have been published by different organizations that try to organize merchandise items in groups with similar features, so that aggregate statistics and analysis can be produced. One of them is the Classification of Products by Activity (CPA), maintained by the European Commission and Eurostat.
According to Eurostat \cite{eurostat2022website}, "\textit{a statistical classification or nomenclature is an exhaustive and structured set of mutually exclusive and well-described categories, often presented in a hierarchy that is reflected by the numeric or alphabetical codes assigned to them, used to standardize concepts and compile statistical data}".
This procedure ensures that data is comparable between EU Member States, and for the purposes of this research, it enables us to put together reports of exchanges declared by different countries.


